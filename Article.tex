\documentclass[times, twoside, watermark]{zHenriquesLab-StyleBioRxiv}
\usepackage{blindtext}

% Please give the surname of the lead author for the running footer
\leadauthor{Denovellis} 

\begin{document}

\title{Categorizing the trajectory dynamics of replay}
\shorttitle{My Template}

% Use letters for affiliations, numbers to show equal authorship (if applicable) and to indicate the corresponding author
\author[1,\Letter]{Eric L. Denovellis}
\author[2, 3]{Anna K. Gillespie}
\author[2, 3]{Michael E. Coulter}
\author[2, 3, 4]{Loren M. Frank}
\author[1]{Uri T. Eden}

\affil[1]{Department of Mathematics and Statistics, Boston University, Boston, Massachusetts}
\affil[2]{Department of Physiology, University of California, San Francisco, San Francisco, California}
\affil[3]{Kavli Institute for Fundamental Neuroscience, University of California, San Francisco, San Francisco, California}
\affil[4]{Howard Hughes Medical Institute, University of California, San Francisco, San Francisco, California}

\maketitle

%TC:break Abstract
%the command above serves to have a word count for the abstract
\begin{abstract}
During sleep and immobility, hippocampal place cells fire in sequences consistent with temporally compressed versions of trajectories previously run by the animal. These replayed sequences are hypothesized to be an important mechanism for the retrieval of spatial memory in service of consolidation and decision-making. Replay events are typically evaluated based on whether they activate sequences of place cells that represent spatially continuous trajectories through the environment, but recent work has shown that these events can have more complex dynamics. For example, sequences can alternate between hovering on a particular spatial location and continuous movement or can represent continuous trajectories in other spatial environments, which may appear spatially incoherent in the context of the current environment.
 
To quantify the structure of replay events, we develop a state space model that uses a combination of discrete and continuous latent states to decompose place cell sequences into categories based on their latent dynamics. Each discrete latent “category” is associated with a type of continuous latent dynamic—such as hovering, spatially fragmented or spatially continuous. This allows for (1) direct comparison between different categories of sequence dynamics, (2) expression of our confidence in one or more categories explaining the data, and (3) characterization of the transitions between categories. In addition, the model can function in 2D, avoiding linearization errors on more complicated environments. We demonstrate the utility of this model on simulated and real data of an animal performing a spatial memory task.

\end {abstract}
%TC:break main
%the command above serves to have a word count for the abstract

\begin{keywords}
Hippocampus | Replay | State Space | Point Process | Switching
\end{keywords}

\begin{corrauthor}
%\texttt{edeno{@}bu.edu}
edeno\at bu.edu
\end{corrauthor}

\section*{Introduction}
\Blindtext

\section*{Results}

Just for kicks here's a citation \cite{Gustafsson2016}. And a reference to a supplement \cref{note:Note1}. And \nameref{note:Note1}.
\Blindtext

\begin{figure}%[tbhp]
\centering
\includegraphics[width=.8\linewidth]{Figures/Figure_1}
\caption{Placeholder image of Iris with a long example caption to show justification setting.}
\label{fig:computerNo}
\end{figure}

\Blindtext

Figure \ref{fig:computerNo} shows an example of how to insert a column-wide figure. To insert a figure wider than one column, please use the \verb|\begin{figure*}...\end{figure*}| environment. Figures wider than one column should be sized to 11.4 cm or 17.8 cm wide. Use \verb|\begin{SCfigure*}...\end{SCfigure*}| for a wide figure with side captions.

\section*{Conclusions}

blablaba \ref{fig:computerNo} 
\blindtext

\subsection*{Blabla} 
\blindtext

\section*{Conclusions}

blablaba \ref{fig:computerNo}
\blindtext

\begin{acknowledgements}
\blindtext
\end{acknowledgements}

\section*{Bibliography}
\bibliography{zHenriquesLab-Mendeley}

%% You can use these special %TC: tags to ignore certain parts of the text.
%TC:ignore
%the command above ignores this section for word count
\onecolumn
\newpage

\section*{Word Counts}
This section is \textit{not} included in the word count. 
\subsection*{Notes on Nature Methods Brief Communication}
\begin{itemize}
\item Abstract: 3 sentences, 70 words.
\item Main text: 3 pages, 2 figures, 1000-1500 words, more figures possible if under 3 pages
\end{itemize}

\subsection*{Statistics on word count}
\detailtexcount
\newpage

%%%%%%%%%%%%%%%%%%%%%%%%%%%%%
% Supplementary Information %
%%%%%%%%%%%%%%%%%%%%%%%%%%%%%
\captionsetup*{format=largeformat}
\section{Something about something} \label{note:Note1} 
\Blindtext

%TC:endignore
%the command above ignores this section for word count

\end{document}
